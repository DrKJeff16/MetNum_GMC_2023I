\documentclass[12pt,letterpaper]{article}

\usepackage[paper=letterpaper]{geometry}

\usepackage[utf8]{inputenc}
\usepackage[T1]{fontenc}
\usepackage{mathptmx}
\usepackage{ragged2e}
\usepackage{pdfpages}
\usepackage{multicol}
\usepackage{authblk}

\usepackage{amsmath}
\usepackage{amssymb}
\usepackage{amsfonts}
\usepackage{physics}
\usepackage{latexsym}
\usepackage{fancyhdr}
\usepackage{setspace}
\usepackage{blindtext}

\usepackage{siunitx}
\sisetup{
	color,
	number-color,
	number-mode=math,
	unit-color,
	unit-mode=math,
	round-mode=places,
	round-precision=5
}

\usepackage{extsizes}
\edef\restoreparindent{\parindent=\the\parindent\relax}
\usepackage{parskip}
\setlength{\parskip}{0.75em}
\restoreparindent
\pagestyle{fancy}

\usepackage{titlesec}
\usepackage{array}
\usepackage{float}
\usepackage{varwidth}
\usepackage[toc,page]{appendix}
\usepackage{kantlipsum}

%\usepackage[labelfont=bf,font={small,it}]{caption}
\usepackage{caption}
\usepackage{subcaption}
\usepackage{booktabs}
\usepackage{multirow}

\usepackage{graphicx}
\usepackage{xcolor}
\usepackage{changepage}
\usepackage{lmodern}

\usepackage{listings}
\usepackage{color}
\renewcommand\lstlistingname{Quelltext}

\lstset{
	language=Python,
	basicstyle=\small\sffamily,
	numbers=left,
	numberstyle=\small,
	frame=tb,
	tabsize=4,
	columns=fixed,
	showstringspaces=true,
	showtabs=true,
	keepspaces,
	commentstyle=\color{Gray},
	keywordstyle=\color{Green}
}

\usepackage{pdftexcmds}
\usepackage{hyperref}
\usepackage{enumitem}

\hypersetup{
	pdftitle={Método De Interpolación De Chebyshev},
	pdfborder=0 0 0,
	destlabel=true,
	colorlinks=true,
	urlcolor=magenta,
	linktoc=section,
	citecolor=blue,
	linkcolor=black,
	filecolor=red,
	pdfpagemode=FullScreen
}

\usepackage{eufrak}

%\usepackage{fontspec}
\usepackage{hyphenat}
\usepackage{longtable}
\usepackage{titletoc}
\usepackage{tocloft}
\usepackage{tocbibind}
\usepackage{afterpage}
%\usepackage[tocgraduated]{tocstyle}
%\usetocstyle{allwithdot}

\usepackage{comment}
\usepackage{csquotes}
\usepackage[spanish,mexico,english,british]{babel}
\usepackage[backend=biber]{biblatex}

\addbibresource{Cuestionario8.bib}
\bibliography{Cuestionario8}

\usepackage{tikz}
\usepackage{pgfplots}
\pgfplotsset{compat=newest}
\usepgfplotslibrary{units}

\DeclareCaptionType{equ}[][]
%\captionsetup[equ]{labelformat=none}

\captionsetup{
	format=hang,
	labelformat=brace,
	labelsep=newline,
	justification=centering,
	width=0.82\textwidth,
	font={
		small,
		it,
		md,
		tt,
		onehalfspacing
	}
}

\author{Guennadi Maximov Cortés}
\date{\today}
\title{Interpolación de Chebyshev}

\renewcommand*\contentsname{{Índice}}

\begin{document}
	\pagenumbering{arabic}
	\maketitle
	\tableofcontents
	\newpage

	\chapter{Introducción} %\label{C1}
		\section{Recurrencia Del Coseno} %\label{C1S1}

		Por las propiedades periódicas de las funciones trigonométricas:

			\begin{figure}
				\begin{center}
					\begin{equation}
						\begin{split}
							\cos\left(2\theta\right) &= \cos^{2}\theta - \sin^2 \theta\\
							&= 2\cos^{2} \theta - 1
						\end{split}
					\end{equation}
				\end{center}
				\label{eq1:cos1}
				\caption{Fórmula resultante cuando ${n=2}$.}
			\end{figure}

			\begin{figure}
				\begin{center}
					\begin{equation}
						\begin{split}
						\cos \left(3\theta\right) &= 4\cos^3 \theta - 3\cos \theta
						\end{split}
					\end{equation}
				\end{center}
				\label{eq2:cos1}
				\caption{Fórmula resultante cuando ${n=3}$.}
			\end{figure}

			En las ecuaciones \ref{eq1:cos1} y \ref{eq2:cos1}, podemos referenciarnos
			a las identidades trigonométricas de los múltiplos de un ángulo ${\theta}$ \cite{TRIGID:1}.

			\begin{figure}
				\begin{center}
					\begin{equation}
						\begin{split}
							\cos \left(n\theta\right) &= \sum_{\text{k par}} \left(-1\right)^{\frac{k}{2}}
							\begin{pmatrix}
								n\\
								k
							\end{pmatrix} \cos^{n-k}\theta\sin^{k}\theta\\
							&= \sum_{i=0}^{\left(n + 1\right) \slash 2} \sum_{j=0}^{i} \left(-1\right)^{i-j}
							\begin{pmatrix}
								n\\
								2i + 1
							\end{pmatrix}
							\begin{pmatrix}
								i\\
								j
							\end{pmatrix} \cos^{n-2\left(i-j\right)-1}\theta
						\end{split}
					\end{equation}
				\end{center}
				\label{gen1:cos1}
				\caption{Forma Discreta de ${\cos\left(n \theta\right)}$}
			\end{figure}

			Si extendemos esto a ${n\in\mathbb{N}}$ para ${\cos\left(x\right)}$,
			tenemos la generalización de la figura \ref{gen1:cos1}\cite{TRIGID:1}.

		\section{Polinomios De Chebyshev} %\label{C1S2}

	%\addtocontents{toc}{\setcounter{tocdepth}{4}}
	\chapter{Interpolación Polinomial} \label{C2}
		\section{Interpolación de Chebyshev} \label{C2S1}

	\medskip

	\printbibliography[title={Referencias}]
	\printbibliography[keyword={YT},heading=subbibintoc,title={Videos de YouTube}]
	\printbibliography[keyword={Web},heading=subbibintoc,title={Páginas Web Y Artículos}]

\end{document}
